\documentclass{article}

\usepackage[english]{babel}

% Set page size and margins
\usepackage[a4paper,top=2cm,bottom=2cm,left=3cm,right=3cm,marginparwidth=1.75cm]{geometry}

% Useful packages
\usepackage{amsmath}
\usepackage{graphicx}
\usepackage{minted}
\usepackage{multirow}
\usepackage[colorlinks=true, allcolors=blue]{hyperref}

\title{\LARGE{\textbf{Algorithm Design 21/22}}\\ \vspace{1cm} Hands On 9 - Game Theory}
\author{Federico Ramacciotti}
\date{}

\begin{document}
\maketitle

\section{Problem}
\begin{enumerate}
    \item After Diabolik's capture, Inspector Ginko is forced to free him because the judge, scared of Eva Kant's revenge, has acquitted him with an excuse. Outraged by the incident, the mayor of Clerville decides to introduce a new legislation to make judges personally liable for their mistakes. The new legislation allows the accused to sue the judge and have him punished in case of error. Consulted on the subject, Ginko is perplexed and decides to ask you to provide him with a formal demonstration of the correctness/incorrectness of this law.
    \item An investment agency wants to collect a certain amount of money for a project. Aimed at convincing all the members of a group of $N$ people to contribute to the fund, it proposes the following contract: each member can freely decide either to contribute with $100$ euros or not to contribute (retaining money on its own wallet). Independently on this choice, after one year, the fund will be rewarded with an interest of $50\%$ and uniformly redistributed among all the $N$ members of the group. Describe the game and find the Nash equilibrium.
\end{enumerate}

\section{Solution}
\subsection{Problem 1}
The game has three players: the judge, the accused and the society (a player with no actions, used just to decide whether the law is correct or not).
The judge can be selfish (only wants to avoid getting sued) or loyal (wants to follow the law and capture the guilty/free the non guilty).
Legend: the order of the utilities is judge, accused and society.

\subsubsection{Case \#1: the accused is guilty}
If the judge is selfish:
\begin{center}
\begin{tabular}{cccc}
                                                     &                              & \multicolumn{2}{c}{\textbf{Accused}}                         \\ \cline{3-4} 
                                                     & \multicolumn{1}{c|}{}        & \multicolumn{1}{c|}{Sue}    & \multicolumn{1}{c|}{Don't Sue} \\ \cline{2-4} 
\multicolumn{1}{c|}{\multirow{2}{*}{\textbf{Judge}}} & \multicolumn{1}{c|}{Capture} & \multicolumn{1}{c|}{0,1; 1} & \multicolumn{1}{c|}{1,0; 2}    \\ \cline{2-4} 
\multicolumn{1}{c|}{}                                & \multicolumn{1}{c|}{Free}    & \multicolumn{1}{c|}{0,0; 0} & \multicolumn{1}{c|}{1,3; 0}    \\ \cline{2-4} 
\end{tabular}
\end{center}
We have a Nash equilibrium in the bottom right cell of the table: the judge doesn't capture the accused and he doesn't sue the judge. Therefore the law is wrong, since the utility for the society is $0$.\\
If the judge is loyal:
\begin{center}
\begin{tabular}{cccc}
                                                     &                              & \multicolumn{2}{c}{\textbf{Accused}}                         \\ \cline{3-4} 
                                                     & \multicolumn{1}{c|}{}        & \multicolumn{1}{c|}{Sue}    & \multicolumn{1}{c|}{Don't Sue} \\ \cline{2-4} 
\multicolumn{1}{c|}{\multirow{2}{*}{\textbf{Judge}}} & \multicolumn{1}{c|}{Capture} & \multicolumn{1}{c|}{2,1; 1} & \multicolumn{1}{c|}{3,0; 2}    \\ \cline{2-4} 
\multicolumn{1}{c|}{}                                & \multicolumn{1}{c|}{Free}    & \multicolumn{1}{c|}{0,0; 0} & \multicolumn{1}{c|}{1,3; 0}    \\ \cline{2-4} 
\end{tabular}
\end{center}
There is no Nash equilibrium.\\
So, if the accused is guilty, the law is wrong.

\subsubsection{Case \#2: the accused is not guilty}
If the judge is selfish:
\begin{center}
\begin{tabular}{cccc}
                                                     &                              & \multicolumn{2}{c}{\textbf{Accused}}                         \\ \cline{3-4} 
                                                     & \multicolumn{1}{c|}{}        & \multicolumn{1}{c|}{Sue}    & \multicolumn{1}{c|}{Don't Sue} \\ \cline{2-4} 
\multicolumn{1}{c|}{\multirow{2}{*}{\textbf{Judge}}} & \multicolumn{1}{c|}{Capture} & \multicolumn{1}{c|}{0,2; 0} & \multicolumn{1}{c|}{1,0; 0}    \\ \cline{2-4} 
\multicolumn{1}{c|}{}                                & \multicolumn{1}{c|}{Free}    & \multicolumn{1}{c|}{0,0; 1} & \multicolumn{1}{c|}{1,3; 2}    \\ \cline{2-4} 
\end{tabular}
\end{center}
We have the same Nash equilibrium as the case \#1, with the difference that now the law is correct, since the society has a utility of $2$.\\
If the judge is loyal:
\begin{center}
\begin{tabular}{cccc}
                                                     &                              & \multicolumn{2}{c}{\textbf{Accused}}                         \\ \cline{3-4} 
                                                     & \multicolumn{1}{c|}{}        & \multicolumn{1}{c|}{Sue}    & \multicolumn{1}{c|}{Don't Sue} \\ \cline{2-4} 
\multicolumn{1}{c|}{\multirow{2}{*}{\textbf{Judge}}} & \multicolumn{1}{c|}{Capture} & \multicolumn{1}{c|}{0,2; 0} & \multicolumn{1}{c|}{1,0; 0}    \\ \cline{2-4} 
\multicolumn{1}{c|}{}                                & \multicolumn{1}{c|}{Free}    & \multicolumn{1}{c|}{2,0; 1} & \multicolumn{1}{c|}{3,3; 2}    \\ \cline{2-4} 
\end{tabular}
\end{center}
Now we have a Nash equilibrium in the bottom right corner and the law is again correct.\\
So, the law is correct if the accused is not guilty.

\subsection{Problem 2}
Let's see an example with 4 people.\\
Legend: P1,P2,P3,P4 are the players, U(P) are the utility functions, Ret is the amount of money that returns to a player after the first year, C means contribute, N means don't contribute.
\begin{center}
\begin{tabular}{|c|c|c|c|c|c|c|c|c|c|c|c|c|c|c|c|c|}
    \hline
    \textbf{-} & \textbf{G} & \textbf{G} & \textbf{G} & \textbf{G} & \textbf{G} & \textbf{G} & \textbf{G} & \textbf{G} & \textbf{G} & \textbf{G} & \textbf{G} & \textbf{G} & \textbf{G} & \textbf{G} & \textbf{G} & \textbf{G} \\ \hline
    \textbf{P1} & C & C & C & C & C & C & C & C & N & N & N & N & N & N & N & N \\ \hline
    \textbf{P2} & C & C & C & C & N & N & N & N & C & C & C & C & N & N & N & N \\ \hline
    \textbf{P3} & C & C & N & N & C & C & N & N & C & C & N & N & C & C & N & N \\ \hline
    \textbf{P4} & C & N & C & N & C & N & C & N & C & N & C & N & C & N & C & N \\ \hline
    \textbf{Ret} & 150 & 112 & 112 & 75 & 112 & 75 & 75 & 37 & 112 & 75 & 75 & 37 & 75 & 37 & 37 & 0 \\ \hline
    \textbf{U(P1)} & 3 & 1 & 1 & -1 & 1 & -1 & -1 & -2 & 5 & 4 & 4 & 2 & 4 & 2 & 2 & 0 \\ \hline
    \textbf{U(P2)} & 3 & 1 & 1 & -1 & 5 & 4 & 4 & 2 & 1 & -1 & -1 & -2 & 4 & 2 & 2 & 0 \\ \hline
    \textbf{U(P3)} & 3 & 1 & 5 & 4 & 1 & -1 & 4 & 2 & 1 & -1 & 4 & 2 & -1 & -2 & 2 & 0 \\ \hline
    \textbf{U(P4)} & 3 & 5 & 1 & 4 & 1 & 4 & -1 & 2 & 1 & 4 & -1 & 2 & -1 & 2 & -2 & 0 \\ 
    \hline
\end{tabular}
\end{center}
There are 5 situations after one year:
\begin{itemize}
    \item if no one contributes, the reward is clearly $0$.
    \item if only 1 person contributes, the reward is $100+100/2=150$ and, per person, is $150/4\cong 37$.
    \item if 2 people contribute, the reward is $200+200/2=300$ and, per person, is $300/4 = 75$.
    \item if 3 people contribute, the reward is $300+300/2=450$ and, per person, is $450/4\cong 112$.
    \item if everyone contributes, the reward is $400+400/2=600$ and, per person, is $600/4 = 150$.
\end{itemize}
The order of preference for each player is:
112 no contribute (+112), 75 no contribute (+75), 150 contribute (+50), 37 no contribute (+37), 112 contribute (+12), 0 no contribute (±0), 75 contribute (-25), 37 contribute (-63).

We have a Nash Equilibrium if no one contributes to the fund. In fact in that situation no one wants to change his choice and contribute, since they have an earning only if at least 3 people contribute and contributing alone has a negative payoff. Moreover, if they are in the situation in which everyone contributes, someone can decide not to contribute in order to increase its earnings.

More in general, let $k$ be the number of people that contribute to the fund and $n$ the total number of players: the return per person is $(k*100+\frac{k*100}{2})/n$. This value implies a positive return only when: $$ \frac{100k+\frac{100k}{2}}{n} \geq 100 \implies 100k+50k \geq 100n \implies 150k \geq 100n \implies k \geq \frac{100}{150}n = \frac{2}{3}n$$
This means that there is a positive payoff (a member earns money, i.e. the next year the return per person is $>100$) only when the contributors are at least $2/3$ of the members. Recalling the example before, there is in fact a positive payoff for a member only if there are more than $\lceil \frac{2}{3}*4 \rceil = 3$ contributors.

\end{document}